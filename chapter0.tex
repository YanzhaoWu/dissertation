%% FRONTMATTER
\begin{frontmatter}

% generate title
\maketitle

\begin{abstract}

With the explosion of data volume and the increasing complexity of data analysis,
large-scale data analysis imposes significant challenges in systems designed for
processing massive datasets. While the current research focuses on scaling out
to a large clusters, these scale-out solutions introduce significant overheads.
This thesis is motivated by the advance of new I/O technologies such as flash
memory. Instead of scaling out, we explore efficient system designs in a single
commodity machine with non-uniform memory architecture (NUMA) and scale to large
datasets by utilizing commodity solid-state drives (SSDs). This raises a question
on how much the new I/O technologies benefit large-scale data analysis. To
address this question, this thesis
develops a data analysis ecosystem called FlashX to target a large range of
data analysis tasks. FlashX includes three subsystems: SAFS, FlashGraph and
FlashMatrix. SAFS is a user-space filesystem optimized for a large SSD array
to deliver maximal I/O throughput from SSDs. FlashGraph is a general-purpose
graph analysis framework that processes graphs in a semi-external memory fashion
and scales to graphs with billions of vertices by utilizing SSDs through SAFS.
FlashMatrix is a matrix-oriented programming framework for general data analysis
with a high-level functional programming interface. Similar to FlashGraph, it
scales matrix operations beyond memory capacity by utilizing SSDs. It executes
R code automatically in parallel and out of core, with a speed comparable to
C code. We demonstrate
that with the current I/O technologies FlashGraph and FlashMatrix in
the (semi-)external-memory mode are capable of achieving performance comparable
to their in-memory mode while significantly outperforming state-of-the-art
in-memory counterparts for a large variety of data analysis tasks.

\vspace{1cm}

\noindent Primary Reader: Some Person\\
Secondary Reader: Someone Else

\end{abstract}

\begin{acknowledgment}

Thanks!

\end{acknowledgment}

\begin{dedication}
 
This thesis is dedicated to \ldots

\end{dedication}

% generate table of contents
\tableofcontents

% generate list of tables
\listoftables

% generate list of figures
\listoffigures

\end{frontmatter}
