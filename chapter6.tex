\chapter{Conclusion}
\label{sec:conclu}
\chaptermark{}

This thesis addresses the challenges in large-scale data analysis using commodity
SSDs. Instead of developing
data analysis algorithms on SSDs directly, we develop programming frameworks
for users to implement complex data analysis algorithms and hides the complexity
of external-memory data analysis and parallel computation.

One of the main contributions of this thesis is
developing a comprehensive data analysis ecosystem called FlashX, which covers
a large range of data analysis tasks. FlashX contains three major subsystems:
SAFS, FlashGraph and FlashMatrix. By seaminglessly integrating these subsystems
together, FlashX achieves efficiency, scalability and generality.

Over the course of studying data analysis tasks on SSDs, we also conclude that
the semi-external memory strategy is a simple but effective way of achieving
both performance and scalability. We adopt semi-external memory in FlashGraph
for graph analysis and in FlashMatrix for sparse matrix multiplication.
In FlashGraph, we keep vertex state in memory and edge lists of graphs on SSDs.
This partitioning enables in-memory vertex communication, the operation
generating majority of small random memory accesses in graph analysis. Thus, we
achieve in-memory performance while scaling beyond memory capacity. We further
extend the concept of semi-external memory to sparse matrix multiplication,
where we keep the sparse matrix on SSDs and the dense matrix or some
columns of the dense matrix in memory. This strategy enables us to achieve
in-memory performance while performing sparse matrix multiplication on a massive
sparse matrix.

By implementing the efficient data analysis ecosystem, we are able to thoroughly
study the role of flash memory in varieties of data analysis at a large scale.
We demonstrate that many graph analysis and machine learning tasks benefit from
flash memory. With careful design and engineering in the data analysis framework
and programming efforts from users, external-memory implementations
of these data analysis tasks achieve performance comparable to that of
state-of-the-art in-memory implementations, while scaling to very large datasets.
This indicates that fast SSDs can replace RAM in data analysis. This potentially
shapes the design of future machines for large-scale data analysis.
