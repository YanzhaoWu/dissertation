\chapter{FlashGraph}
\label{sec:fg}
\chaptermark{FlashGraph: Processing Billion-Node Graphs on an Array of
Commodity SSDs}

This chapter describes FlashGraph, a general-purpose graph processing framework
for massive graphs. FlashGraph builds on top of SAFS to utilize commodity SSDs.
We demonstrate that FlashGraph enables a multicore server to process graphs
with billions of vertices and hundreds of billions of edges, with minimal
performance loss. FlashGraph
processes graphs in a semi-external memory fashion, i.e., it stores vertex state
in memory and edge lists on SSDs. It hides latency by overlapping computation
with I/O. To save I/O bandwidth, FlashGraph only accesses edge lists requested
by applications from SSDs; to increase I/O throughput and reduce CPU overhead for I/O,
it conservatively merges I/O requests. These designs maximize performance
for applications with different I/O characteristics. FlashGraph exposes
a general and flexible vertex-centric programming interface that can express
a wide variety of graph algorithms and their optimizations. We demonstrate that
FlashGraph in semi-external memory performs many algorithms with performance
up to 80\% of its in-memory implementation and significantly outperforms
PowerGraph, a popular distributed in-memory graph engine.

\section{Introduction}
\input{FlashGraph/intro}

\section{Related Work}
\input{FlashGraph/relwork}

\input{FlashGraph/design}

\input{FlashGraph/eval}

\section{Conclusions}
\input{FlashGraph/conclusion}
